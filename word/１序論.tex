% 近年, CTスキャナーは小型化,高精度化され,医療業界を中心に頻繁に活用されている.
% その中で古文献の解読にCTスキャン技術を用いる研究が現れるようになった[].
% それらは主に,経年劣化による構造的な脆さや学術的希少価値を理由に開くことができなくなった書物の内部を非侵襲的に測定し,可視化するといったものである.

歴史的な文献には過去の出来事に関する貴重な記述が含まれており, これを解読することで歴史を紐解く手がかりを得ることができる.
こうした文献は歴史学研究における根幹的な資料であり、歴史学者達にとって重要な意味を持つ.
しかし, 経年劣化や外的な浸食, 文書の重要性などの理由から,古文献の中には文章を読むことができないものも多く存在する.
開封不可能な手紙, 炭化した文献などがこれに該当する.
こうした古文献を読むためには, 非侵襲的な測定法によって内部に情報を取得, 可視化する必要がある.

1750年,イタリア・カンパニア州において Karl Weber はヘラクラネウム付近にて, 紀元79年のヴェスヴィオ火山噴火によって炭化した1800枚を超えるパピルス紙を発見した[].
このパピルス紙は筒状に巻かれており,損傷させずに開くことは困難であったため, 非侵襲的な手法を用いらなければ解読が不可能であった.
また, 洪水や火災といった外的な侵襲により文献が開くことができなくなる例としては, ドイツのDuchess Anna Amalia Library での2004年の火災が挙げられる.
この火災では62000を超える古文献が,火災による高温と焼失,さらに消火の際の水によって重大な被害を被った[].

日本においても,歴史的な文献が水害により固着し、読むことができなくなった例が存在する.
またそうした外的な要因で閲覧が不可能になったもの以外にも, 本の綴じ目近くの「のど」と呼ばれる箇所に綴じ込まれている箇所,紙背文書と呼ばれる袋綴じ状態となった箇所など, 文献を損傷させずに読むことが難しいものがある.
こうした文献中の文章を取得する手法は未だ確立されていない.
元の文書が修復される際に副次的に調査することが可能になることもあるが, 研究者が日常的に文献を調査することは現状不可能である.

非侵襲的な文献閲覧手法としては, 3次元X線CTイメージングが最も広く普及している.
この手法は従来医療の分野で多く用いられてきたが, 近年では世界文化遺産などの(何?建物or本orその他を明確に)調査にも用いられている[].
先行研究から,マイクロCTシステムが古文献の調査に有効であることがわかっている[].
%マイクロCTシステムの原理
インクと紙中のセルロースの間に,X線の減衰率の差が少しでも存在すれば, CTイメージングの測定結果に一定のコントラストとして現れる.
十分な解像度を持った3次元X線CTスキャナーであれば,古文献の内容を非侵襲的に取得し, デジタル化することが可能である.
しかし, この手法は測定時に対象が放射線に曝露されるため,経年劣化がスピードを加速してしまうという欠点を持つ.
%測定という言葉が本当にあっているか CTであれば撮影でないか?

3次元テラヘルツイメージングやX線位相イメージングを用いれば, 測定対象に放射線を浴びせずにデジタル化することができる.
前者は, 〜〜〜〜〜であり後者は〜〜〜〜〜〜というものである. %テラヘルツイメージングとx線位相イメージングの原理を説明
しかしテラヘルツイメージングでは, テラヘルツ波の減衰が顕著なため, 数ページほどしか文献データを取得することができない.
X線位相イメージングは測定装置が非常に大きく持ち運べないため, 図書館等に保管される貴重な文献を調査する際には文献を特定の研究施設に持ち運ぶ必要があり, コストやセキュリティの面で問題がある.
マイクロCTスキャナーは,十分な深度まで文献を測定することが可能であり,また測定機器もX線位相イメージングよりも小さいため,古文献の可視化に現時点でもっとも有効な手段と考えられる.

CTイメージングを冊子体に対して行うと、3次元のボリュームデータが得られる.
冊子体中の文章を読むためには, このデータから文章を抽出する必要がある.
従来, 研究者達はデータから文章を抽出するために, ボクセルデータの中から冊子の各ページを手動で選択する必要があった(要参考文献).
以降, この作業をアノテーションと呼ぶ.
アノテーションは通常手動で行われるため, 文献の内容抽出の高速化,大規模化に対するボトルネックとなっている.
ボリュームデータに含まれる全てのページにアノテーションを行うには, 膨大な時間が必要となるため効率的でない.

そこで本研究では,ラプラス方程式を用いたアノテーションの補間手法を提案し, マイクロCTスキャナーから得られた冊子体ボリュームデータからの文章抽出の効率化を目指す..
これは冊子体ボリュームデータを一種の積層コンデンサとみなしラプラス方程式を用いて計算をすることで,アノテーション数を減らす手法である.
必要なアノテーション数が減少すれば, 非侵襲的な手法で冊子体から文章を抽出するのが容易になるであろう.
また, 提案手法の有効性を確かめるため,CTスキャンにより冊子状ボリュームデータを取得し, 本手法を適用した結果を定量的に評価する.
鉄分を含むインクを用いて作成した擬似的な古代文献(以降,擬似試料[要修正]と呼ぶ.)を用意し,この擬似試料をCTでスキャンし冊子体ボリュームデータを取得する.
このボリュームデータに完全に人の手でアノテーションを施すものと, 提案手法を適用するものとで、その文章抽出結果を比較し、手法の有効性を議論する.

%仮説の説明は導入では冗長

% 本研究において,仮説は以下のように設定する.
% H1 冊子体ボリュームデータからのページ抽出課題において,ページ番号を推定する関数がラプラス方程式を満たすと仮定し,関数のパラメータを最適化すれば,ページ抽出の精度を保ちながら,必要なアノテーション数を減らすことを可能にする.

% H1.1 冊子体ボリュームデータからのページ抽出課題において,ページ番号を推定する関数がラプラス方程式を満たすと仮定し,関数のパラメータを最適化すれば,ページ抽出の精度を保ちながら,1ページあたりに必要なアノテーション数を減らすことを可能にする.

% H1.2 冊子体ボリュームデータからのページ抽出課題において,ページ番号を推定する関数がラプラス方程式を満たすと仮定し,関数のパラメータを最適化すれば,ページ抽出の精度を保ちながら,アノテーションが必要なページ数を減らすことを可能にする.

% H1.3 冊子体ボリュームデータからのページ抽出課題において,ページ番号を推定する関数がラプラス方程式を満たすと仮定し,関数のパラメータを最適化すれば,ページ抽出の精度を保ちながら,1ページあたりに必要なアノテーション数とアノテーションが必要なページ数を同時に減らすことを可能にする.

% H2 本手法で用いるラプラス方程式に適切な補正項を加えることで,冊子体ボリュームデータのページ分布をより精度よく推定できて,アノテーション数をさらに削減する
% [間に合うか不明]

本手法では用いるラプラス方程式のパラメータ推定にLevenerg-Marquardt法による非線形最小二乗法を用いる.
非線形最小二乗法は,最小二乗法を非線形関数のパラメータフィッティングへ拡張したもので,目的変数と説明変数の間の関係を表す式をサンプルデータとモデル関数の残差平方和の最小化によって得る方法である.
本研究では,数値解析ソフトウエア SciPy に実装されている非線形最小二乗法の関数を用いる.

%手法を提案する研究の場合、メインは手法の質
%有効性を評価するために検証は必要だが、3章使って検証してるのはあまりに変
%「実験」という言葉を使うとなんの研究かわからなくなる、あくまでこれは提案手法の有効性評価のパートであり、これがウェイトをしめるのはとてもよくない

本稿の以降の構成は次のとおりである. まず第 2 章では関連研究について説明する.
続いて第 3 章で冊子体ボリュームデータについて述べ,第 4 章では提案手法について詳細に説明する.
%以降要改善
第 5 章では, 仮説と仮説の検証方法を述べる.
第 6 章では実験結果について述べ,第 7 章では実験結果の考察をする.
最後に第 8 章で結論を述べる.



関連研究1
本章では関連研究について記述する.
2.1節では
CTスキャナーによるボリュームデータの取得について
ボリュームデータからの平面抽出
について説明する.
2.2節では
等値平面(以降,ISO Surfaceと呼ぶ)の可視化について述べる.
2.3節では
非線形方程式のパラメータ推定について記述する.
2.4節では
OCRにおける文字認識について述べる.

ボリュームデータからの平面抽出とその可視化について
CTスキャナーなど3次元測定装置によって得られたボリュームデータから平面抽出する方法には複数の手法が存在する.
・人間のアノテーション
・ISO Surface

CTスキャナーについて





非線形関数のパラメータ推定について
まず線形関数のパラメータ推定についてであるが,もっとも古典的な手法として最小二乗法があげられる.想定する関数をfとして,既知の関数


・ボリュームデータからのページ抽出
・ISO Surface の可視化について
・非線形方程式のパラメータ推定について
・OCRについて
・CTスキャナーによる紙のスキャンについて


